%%%%%%%%%%%%%%%%%
% This is an example CV created using altacv.cls (v1.6, 21 May 2021) written by
% Alejandro Vega Gómez (alexvega48@gmail.com), based on the
% Cv created by BusinessInsider at http://www.businessinsider.my/a-sample-resume-for-marissa-mayer-2016-7/?r=US&IR=T
%
%% It may be distributed and/or modified under the
%% conditions of the LaTeX Project Public License, either version 1.3
%% of this license or (at your option) any later version.
%% The latest version of this license is in
%%    http://www.latex-project.org/lppl.txt
%% and version 1.3 or later is part of all distributions of LaTeX
%% version 2003/12/01 or later.
%%%%%%%%%%%%%%%%

%% Use the "normalphoto" option if you want a normal photo instead of cropped to a circle
\documentclass[10pt,a4paper,normalphoto,hidelinks]{altacv}

%\documentclass[10pt,a4paper,ragged2e,withhyper]{altacv}

%% AltaCV uses the fontawesome5 package.
%% See http://texdoc.net/pkg/fontawesome5 for full list of symbols.

% Change the page layout if you need to
\geometry{left=1.25cm,right=1.25cm,top=1.5cm,bottom=1.5cm,columnsep=1.2cm}

% The paracol package lets you typeset columns of text in parallel
\usepackage{paracol}
%\usepackage[hidelinks]{hyperref}
\usepackage{hyperref}


% Change the font if you want to, depending on whether
% you're using pdflatex or xelatex/lualatex
\ifxetexorluatex
  % If using xelatex or lualatex:
  \setmainfont{Lato}
\else
  % If using pdflatex:
  \usepackage[default]{lato}
\fi

% Change the colours if you want to

\definecolor{VividBlue}{HTML}{0012E9}
\definecolor{VividPurple}{HTML}{3E0097}
\definecolor{SlateGrey}{HTML}{2E2E2E}
\definecolor{LightGrey}{HTML}{666666}
% \colorlet{name}{black}
% \colorlet{tagline}{PastelRed}
\colorlet{heading}{VividBlue}
\colorlet{headingrule}{VividBlue}
% \colorlet{subheading}{PastelRed}
\colorlet{accent}{VividBlue}
\colorlet{emphasis}{SlateGrey}
\colorlet{body}{LightGrey}

% Change some fonts, if necessary
% \renewcommand{\namefont}{\Huge\rmfamily\bfseries}
% \renewcommand{\personalinfofont}{\footnotesize}
% \renewcommand{\cvsectionfont}{\LARGE\rmfamily\bfseries}
% \renewcommand{\cvsubsectionfont}{\large\bfseries}

% Change the bullets for itemize and rating marker
% for \cvskill if you want to
\renewcommand{\itemmarker}{{\small\textbullet}}
\renewcommand{\ratingmarker}{\faCircle}

%% Use (and optionally edit if necessary) this .cfg if you
%% want to use an author-year reference style like APA(6)
%% for your publication list
% When using APA6 if you need more author names to be listed
% because you're e.g. the 12th author, add apamaxprtauth=12
\usepackage[backend=biber,style=apa6,sorting=ydnt]{biblatex}
\defbibheading{pubtype}{\cvsubsection{#1}}
\renewcommand{\bibsetup}{\vspace*{-\baselineskip}}
\AtEveryBibitem{\makebox[\bibhang][l]{\itemmarker}}
\setlength{\bibitemsep}{0.25\baselineskip}
\setlength{\bibhang}{1.25em}


%% Use (and optionally edit if necessary) this .cfg if you
%% want an originally numerical reference style like IEEE
%% for your publication list
% \usepackage[backend=biber,style=ieee,sorting=ydnt]{biblatex}
%% For removing numbering entirely when using a numeric style
\setlength{\bibhang}{1.25em}
\DeclareFieldFormat{labelnumberwidth}{\makebox[\bibhang][l]{\itemmarker}}
\setlength{\biblabelsep}{0pt}
\defbibheading{pubtype}{\cvsubsection{#1}}
\renewcommand{\bibsetup}{\vspace*{-\baselineskip}}


%% sample.bib contains your publications
\addbibresource{sample.bib}

\begin{document}
\name{Alejandro Vega Gómez}
\tagline{Embedded Software Engineer looking for new challenges}
% Cropped to square from https://en.wikipedia.org/wiki/Marissa_Mayer#/media/File:Marissa_Mayer_May_2014_(cropped).jpg, CC-BY 2.0
%% You can add multiple photos on the left or right
%\photoR{2.5cm}{alejandro_vega_gomez}
% \photoL{2cm}{Yacht_High,Suitcase_High}
\personalinfo{%
  % Not all of these are required!
  % You can add your own with \printinfo{symbol}{detail}
  \email{alexvega48@gmail.com}
	\phone{+34 626 88 98 46}
  %\mailaddress{Calle Ginzo de Limia 26}
  \location{Madrid, España}
  
  \linkedin{}
  \url{https://www.linkedin.com/in/alejandrovegagomez/}
  \quad  
  \github{ } % I'm just making this up though.
  \url{https://github.com/aavg-dev}
  % \gitlab{your_id}
%   \orcid{0000-0000-0000-0000} % Obviously making this up too.
  %% You can add your own arbitrary detail with
  %% \printinfo{symbol}{detail}[optional hyperlink prefix]
  % \printinfo{\faPaw}{Hey ho!}
  %% Or you can declare your own field with
  %% \NewInfoFiled{fieldname}{symbol}[optional hyperlink prefix] and use it:
  % \NewInfoField{gitlab}{\faGitlab}[https://gitlab.com/]
  % \gitlab{your_id}
	%%
  %% For services and platforms like Mastodon where there isn't a
  %% straightforward relation between the user ID/nickname and the hyperlink,
  %% you can use \printinfo directly e.g.
  % \printinfo{\faMastodon}{@username@instace}[https://instance.url/@username]
  %% But if you absolutely want to create new dedicated info fields for
  %% such platforms, then use \NewInfoField* with a star:
  % \NewInfoField*{mastodon}{\faMastodon}
  %% then you can use \mastodon, with TWO arguments where the 2nd argument is
  %% the full hyperlink.
  % \mastodon{@username@instance}{https://instance.url/@username}
}

\makecvheader

%% Depending on your tastes, you may want to make fonts of itemize environments slightly smaller
\AtBeginEnvironment{itemize}{\small}

%% Set the left/right column width ratio to 6:4.
\columnratio{0.6}

% Start a 2-column paracol. Both the left and right columns will automatically
% break across pages if things get too long.
\begin{paracol}{2}

\cvsection{Experience}

\cvevent{Embedded Software Engineer}{GMV}{Jan 2017 - Now}{Madrid - Valladolid, ES}
\begin{itemize}
\item Conception, design and development of Deepsy, a multi-platform framework created for Intelligent Transport Systems (C++)
\item Main developer for telecommunication (Modem) and geopositioning (GNSS) services . Development of application layer, API and tool.  
\item Development of services with different purposes: OTA updates, processes control, automatic report of device status, Audio service, RTC service. 

\item C++: STD and Boost libraries, specialized in Asio and Thread. Use of other libraries such as Signal,Spirit,Regex,Filesystem. Inter-process communication libraries (ZMQ) and serialization (Protobuf, JSON).
\item Scrum + Agile: Experience as Scrum Master (1,5 years) and responsible for Product Owner tasks. 
\item CI/CD: Integration of a Software Quality Gate for metric analysis and code quality using Sonarqube.
\item Organization of demos for external and internal clients and organization of workshops to create new applications and functionalities. 

\end{itemize}

\divider

\cvevent{Software Engineer }{Nestwork}{Jan 2015 - Jan 2017}{Madrid, ES}
\begin{itemize}
\item Development of applications for consumer electronics: personal security wearables and pet wearables.
\item Behaviour Driven Development (BDD) using Gherkin and Cucumber. 

\item Investigation of new technologies and creation intellectual property for the company (I+D)
\item Testing and Validation of prototypes at different levels: industrial design - casing and PCB. 
\end{itemize}


% \divider
\cvsection{Courses and Events}

\cvevent{Linux Drivers }{} {} {TimeSys, Pensilvania}


\cvevent{Interactive programming in Python 
}{}{}{Rice University, Houston}


\cvevent{Agile Methodologies}{} {} {}
\begin{itemize}
\item Agile Meets Design Thinking [University of Virginia]
\item Managing an Agile Team [University of Virginia]
\item Running Product Design Sprints [University of Virginia]
\item Testing with Agile  [University of Virginia]
\item Product Owner [233 Grados de TI]

\end{itemize}

\cvevent{Oratory and how to talk in public}{}{}{Habilitips, Madrid}



% use ONLY \newpage if you want to force a page break for
% ONLY the currentc column
\newpage

%\cvsection{Publications}

%\nocite{*}

%\printbibliography[heading=pubtype,title={\printinfo{\faBook}{Books}},type=book]

%\divider

%\printbibliography[heading=pubtype,title={\printinfo{\faFile*[regular]}{Journal Articles}}, type=article]

%\divider

%\printbibliography[heading=pubtype,title={\printinfo{\faUsers}{Conference Proceedings}},type=inproceedings]

%% Switch to the right column. This will now automatically move to the second
%% page if the content is too long.
\switchcolumn

\cvsection{Achievements}

\cvachievement{\faHeartbeat}{Commitment and  y persistence}{Shown during the years within the company}

\cvachievement{\faChartLine}{Personal growth and GMV growth}{After the succesful evolution of GMV-ITS}


%\divider

%\cvachievement{\faChartLine}{Google's Growth}{from a hundred thousand searches per day to over a billion}

%\divider

%\cvachievement{\faFemale}{Inspiring women in tech}{Youngest CEO on Fortune's list of 50 most powerful women}

\cvsection{Soft Skills}

\cvtag{Critical Thinking}
\cvtag{Adaptation}
\cvtag{Learning capacity}
\cvtag{Empathy}
\cvtag{Personal motivation}
\cvtag{Active and decisive}


\cvsection{Hard skills}

\cvtag{C++}
\cvtag{Python}
\cvtag{Linux}
\cvtag{Gtest} 
\cvtag{Gmock}
\cvtag{Git}
\cvtag{Qt}
\cvtag{Bash}
\cvtag{Yocto}
\cvtag{Dev-Ops}
\cvtag{Robotest}
\cvtag{CI/CD}
\cvtag{Regex}
\cvtag{Multithreading}
\cvtag{Atlassian tools}



%\divider\smallskip
\cvsection{Interests}
 	
\cvtag{Agile methodologies}
\cvtag{UX}
\cvtag{UI}
\cvtag{Software Product}
\cvtag{Renewable energies}
\cvtag{Autonomous vehicle}
\cvtag{Volunteering}

\cvsection{Languages}

\cvskill{Spanish (Native)}{5}
\cvskill{English}{4}
% \divider

% \divider

\cvskill{German}{2.5} %% supports X.5 values.

\cvsection{Education}

\cvevent{Master's Degree in Industrial Engineering}{University Carlos III de Madrid}{2014 - 2016}{}

\cvevent{Bilingual Engineering Degree in Industrial Technologies}{University Carlos III de Madrid}{2010 - 2014}{}

\cvevent{Erasmus experience in Poland}{University Carlos III de Madrid}{2014}{}


%\cvsection{Voluntariado}
%\cvevent{Monitor de campamentos}{}{}{}

%\newpage

%\cvsection{Referees}

% \cvref{name}{email}{mailing address}
%\cvref{Prof.\ Alpha Beta}{Institute}{a.beta@university.edu}
%{Address Line 1\\Address line 2}

%\divider

%\cvref{Prof.\ Gamma Delta}{Institute}{g.delta@university.edu}
%{Address Line 1\\Address line 2}

\end{paracol}

\end{document}
