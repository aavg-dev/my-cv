%%%%%%%%%%%%%%%%%
% This is an example CV created using altacv.cls (v1.6, 21 May 2021) written by
% Alejandro Vega Gómez (alexvega48@gmail.com), based on the
% Cv created by BusinessInsider at http://www.businessinsider.my/a-sample-resume-for-marissa-mayer-2016-7/?r=US&IR=T
%
%% It may be distributed and/or modified under the
%% conditions of the LaTeX Project Public License, either version 1.3
%% of this license or (at your option) any later version.
%% The latest version of this license is in
%%    http://www.latex-project.org/lppl.txt
%% and version 1.3 or later is part of all distributions of LaTeX
%% version 2003/12/01 or later.
%%%%%%%%%%%%%%%%

%% Use the "normalphoto" option if you want a normal photo instead of cropped to a circle
\documentclass[10pt,a4paper,normalphoto]{altacv}

%\documentclass[10pt,a4paper,ragged2e,withhyper]{altacv}

%% AltaCV uses the fontawesome5 package.
%% See http://texdoc.net/pkg/fontawesome5 for full list of symbols.

% Change the page layout if you need to
\geometry{left=1.25cm,right=1.25cm,top=1.5cm,bottom=1.5cm,columnsep=1.2cm}

% The paracol package lets you typeset columns of text in parallel
\usepackage{paracol}


% Change the font if you want to, depending on whether
% you're using pdflatex or xelatex/lualatex
\ifxetexorluatex
  % If using xelatex or lualatex:
  \setmainfont{Lato}
\else
  % If using pdflatex:
  \usepackage[default]{lato}
\fi

% Change the colours if you want to

\definecolor{VividBlue}{HTML}{0012E9}
\definecolor{VividPurple}{HTML}{3E0097}
\definecolor{SlateGrey}{HTML}{2E2E2E}
\definecolor{LightGrey}{HTML}{666666}
% \colorlet{name}{black}
% \colorlet{tagline}{PastelRed}
\colorlet{heading}{VividBlue}
\colorlet{headingrule}{VividBlue}
% \colorlet{subheading}{PastelRed}
\colorlet{accent}{VividBlue}
\colorlet{emphasis}{SlateGrey}
\colorlet{body}{LightGrey}

% Change some fonts, if necessary
% \renewcommand{\namefont}{\Huge\rmfamily\bfseries}
% \renewcommand{\personalinfofont}{\footnotesize}
% \renewcommand{\cvsectionfont}{\LARGE\rmfamily\bfseries}
% \renewcommand{\cvsubsectionfont}{\large\bfseries}

% Change the bullets for itemize and rating marker
% for \cvskill if you want to
\renewcommand{\itemmarker}{{\small\textbullet}}
\renewcommand{\ratingmarker}{\faCircle}

%% Use (and optionally edit if necessary) this .cfg if you
%% want to use an author-year reference style like APA(6)
%% for your publication list
% When using APA6 if you need more author names to be listed
% because you're e.g. the 12th author, add apamaxprtauth=12
\usepackage[backend=biber,style=apa6,sorting=ydnt]{biblatex}
\defbibheading{pubtype}{\cvsubsection{#1}}
\renewcommand{\bibsetup}{\vspace*{-\baselineskip}}
\AtEveryBibitem{\makebox[\bibhang][l]{\itemmarker}}
\setlength{\bibitemsep}{0.25\baselineskip}
\setlength{\bibhang}{1.25em}


%% Use (and optionally edit if necessary) this .cfg if you
%% want an originally numerical reference style like IEEE
%% for your publication list
% \usepackage[backend=biber,style=ieee,sorting=ydnt]{biblatex}
%% For removing numbering entirely when using a numeric style
\setlength{\bibhang}{1.25em}
\DeclareFieldFormat{labelnumberwidth}{\makebox[\bibhang][l]{\itemmarker}}
\setlength{\biblabelsep}{0pt}
\defbibheading{pubtype}{\cvsubsection{#1}}
\renewcommand{\bibsetup}{\vspace*{-\baselineskip}}


%% sample.bib contains your publications
\addbibresource{sample.bib}

\begin{document}
\name{Alejandro Vega Gómez}
\tagline{Ingeniero de Software Embebido con ganas de asumir nuevos retos}
% Cropped to square from https://en.wikipedia.org/wiki/Marissa_Mayer#/media/File:Marissa_Mayer_May_2014_(cropped).jpg, CC-BY 2.0
%% You can add multiple photos on the left or right
%\photoR{2.5cm}{alejandro_vega_gomez}
% \photoL{2cm}{Yacht_High,Suitcase_High}
\personalinfo{%
  % Not all of these are required!
  % You can add your own with \printinfo{symbol}{detail}
  \email{alexvega48@gmail.com}
	\phone{+34 626 88 98 46}
  %\mailaddress{Calle Ginzo de Limia 26}
  \location{Madrid, España}
  \linkedin{www.linkedin.com/in/alejandro-vega-gómez-137585b5}
  \github{github.com/} % I'm just making this up though.
%   \orcid{0000-0000-0000-0000} % Obviously making this up too.
  %% You can add your own arbitrary detail with
  %% \printinfo{symbol}{detail}[optional hyperlink prefix]
  % \printinfo{\faPaw}{Hey ho!}
  %% Or you can declare your own field with
  %% \NewInfoFiled{fieldname}{symbol}[optional hyperlink prefix] and use it:
  % \NewInfoField{gitlab}{\faGitlab}[https://gitlab.com/]
  % \gitlab{your_id}
	%%
  %% For services and platforms like Mastodon where there isn't a
  %% straightforward relation between the user ID/nickname and the hyperlink,
  %% you can use \printinfo directly e.g.
  % \printinfo{\faMastodon}{@username@instace}[https://instance.url/@username]
  %% But if you absolutely want to create new dedicated info fields for
  %% such platforms, then use \NewInfoField* with a star:
  % \NewInfoField*{mastodon}{\faMastodon}
  %% then you can use \mastodon, with TWO arguments where the 2nd argument is
  %% the full hyperlink.
  % \mastodon{@username@instance}{https://instance.url/@username}
}

\makecvheader

%% Depending on your tastes, you may want to make fonts of itemize environments slightly smaller
\AtBeginEnvironment{itemize}{\small}

%% Set the left/right column width ratio to 6:4.
\columnratio{0.6}

% Start a 2-column paracol. Both the left and right columns will automatically
% break across pages if things get too long.
\begin{paracol}{2}

\cvsection{Experiencia}

\cvevent{Ingeniero de Software Embebido}{GMV}{Enero 2017 - Actualidad}{Valladolid - Madrid, ES}
\begin{itemize}
\item Diseño, concepción y desarrollo de Deepsy, un framework multiplataforma  para Sistemas Inteligentes del Transporte (C++).
\item Principal desarrollador de servicios de telecomunicación (Modem) y geoposicionamiento (GNSS). Desarrollo de aplicación y API.
\item Desarrollo de servicios para distintos propósitos: actualizaciones OTA, control de procesos, reporte automático del estado del equipo, servicio de Audio, servicio RTC. 
\item C++: Librerías STD y Boost, especialmente Asio y Thread. Uso de otras librerías como Signal,Spirit,Regex,Filesystem. Librerías de comunicación entre procesos (ZMQ) y  serialización (Protobuf, JSON).
\item Scrum + Agile: Experiencia como Scrum Máster (1,5 años) y desempeño de tareas de Product Owner
\item CI/CD: Integración de una Quality Gate Software para análisis de métricas y calidad del código usando Sonarqube.
\item Demostraciones para clientes internos y externos, organización de workshops para creación de nuevas aplicaciones y funcionalidades

\end{itemize}

\divider

\cvevent{Ingeniero de Software }{Nestwork}{Enero 2015 -- Enero 2017}{Madrid, ES}
\begin{itemize}
\item Desarrollo de aplicaciones para dispositivos de electrónica de consumo: wearables de seguridad personal y para mascotas. 
\item Desarrollo orientado a comportamiento (BDD) usando Gherkin y Cucumber.
\item Investigación de nuevas tecnologías y creación de propiedad intelectual (I+D)
\item Testing y validación de prototipos en varios niveles: diseño industrial , casing y PCB 
\end{itemize}


% \divider
\cvsection{Cursos y eventos}

\cvevent{Linux Drivers}{Timesys} {} {}

\cvevent{Cursos en Agilidad}{Coursera.org} {} {}
\begin{itemize}
\item Agile Meets Design Thinking
\item Managing an Agile Team
\item Running Product Design Sprints
\item Testing with Agile 
\end{itemize}

\cvevent{Hablar en público - Ambassador Program }{Habilitips}{}{}


% use ONLY \newpage if you want to force a page break for
% ONLY the currentc column
\newpage

%\cvsection{Publications}

%\nocite{*}

%\printbibliography[heading=pubtype,title={\printinfo{\faBook}{Books}},type=book]

%\divider

%\printbibliography[heading=pubtype,title={\printinfo{\faFile*[regular]}{Journal Articles}}, type=article]

%\divider

%\printbibliography[heading=pubtype,title={\printinfo{\faUsers}{Conference Proceedings}},type=inproceedings]

%% Switch to the right column. This will now automatically move to the second
%% page if the content is too long.
\switchcolumn

\cvsection{Logros}

\cvachievement{\faHeartbeat}{Compromiso y persistencia}{Mostrado con el paso de los años en la empresa}

\cvachievement{\faChartLine}{Crecimiento personal y de GMV}{Tras la exitosa evolución de GMV-ITS}


%\divider

%\cvachievement{\faChartLine}{Google's Growth}{from a hundred thousand searches per day to over a billion}

%\divider

%\cvachievement{\faFemale}{Inspiring women in tech}{Youngest CEO on Fortune's list of 50 most powerful women}

\cvsection{Soft Skills}

\cvtag{Motivación personal}
\cvtag{Activo y resolutivo}
\cvtag{Trabajador}
\cvtag{Empático}
\cvtag{Adaptación y aprendizaje rápidos}


\cvsection{Hard skills}

\cvtag{C++}
\cvtag{Python}
\cvtag{Linux}
\cvtag{Gtest} 
\cvtag{Gmock}
\cvtag{Git}
\cvtag{Qt}
\cvtag{Bash}
\cvtag{Yocto}
\cvtag{Dev-Ops}
\cvtag{Robotest}
\cvtag{CI/CD}
\cvtag{Regex}
\cvtag{Multithreading}
\cvtag{Atlassian tools}



%\divider\smallskip
\cvsection{Intereses}
 	
\cvtag{Metodologías ágiles}
\cvtag{UX}
\cvtag{UI}
\cvtag{Interacción con cliente}
\cvtag{Gestión de Producto}
\cvtag{Voluntariado}


\cvsection{Idiomas}

\cvskill{Español (Nativo)}{5}
\cvskill{Inglés}{4}
% \divider

% \divider

\cvskill{Alemán}{2.5} %% supports X.5 values.

\cvsection{Educación}

\cvevent{Máster en Ingeniería Industrial}{Universidad Carlos III de Madrid}{2014 - 2016}{}

\cvevent{Grado Bilingüe en Ingeniería de Tecnologías Industriales}{Universidad Carlos III de Madrid}{2010 - 2014}{}

\cvevent{Experiencia Erasmus en Polonia}{Universidad Carlos III de Madrid}{2014}{}


%\cvsection{Voluntariado}
%\cvevent{Monitor de campamentos}{}{}{}

%\newpage

%\cvsection{Referees}

% \cvref{name}{email}{mailing address}
%\cvref{Prof.\ Alpha Beta}{Institute}{a.beta@university.edu}
%{Address Line 1\\Address line 2}

%\divider

%\cvref{Prof.\ Gamma Delta}{Institute}{g.delta@university.edu}
%{Address Line 1\\Address line 2}

\end{paracol}

\end{document}
